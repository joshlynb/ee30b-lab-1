\documentclass{article}
\usepackage{float}
\usepackage{graphicx} % Required for inserting images
\usepackage{siunitx}
\usepackage{hyperref}
\usepackage{gensymb}
\usepackage{fixltx2e}
\usepackage{textgreek}
\title{EE30B Lab 1: Impedance Measurements}
\author{Joshlyn Bui: [INSERT SID] \\ Brittany Chan: 862328958 \\ Aishwarya Karra: [INSERT SID]}
\date{January 2026}

\begin{document}
\maketitle

\section{Introduction}
\article The goals of this lab are to understand impedance characterization, understand the experimental methods of impedance measurements, and verifying capacitor impedance as a function of frequency. We achieved these goals by an oscilloscope and a function generator that produces a sinusoidal output of amplitude 5V. By using the MATH function in the oscilloscope and subtracting the waveforms of the difference between voltages of CH 1 (function generator) and CH 2 (peak-to-peak capacitor voltage), we obtain the capacitor impedance.

\section{Theory}

\section{Prelab}
\begin{enumerate}
    \item How many parameters are needed to completely describe impedance? What are they?
    \begin{itemize}
        \item 2 parameters are need to describe impedance, and they are resistance (R) and reactance (X). 
    \end{itemize}

    \item If $Z = 3 + j4$ for some circuit element, what is the phase angle between its voltage and current? Is its current lagging or leading voltage? 
    \begin{itemize}
        \item 
    \end{itemize}

    \item Could we connect the oscilloscope probe across the resistor in \textbf{Figure 4} to measure its voltage waveform $v_{r}(t)$ instead of using the \textbf{MATH} function of the oscilloscope? Why?
    \begin{itemize}
        \item 
    \end{itemize}
    
\end{enumerate}
\section{Design Calculations and Circuit Schematic}


\section{Experimental Data}

\section{Data Analysis}

\section{ Practical Problems Encountered and Resolved}


\section{Conclusions}


\end{document}


\end{document}
